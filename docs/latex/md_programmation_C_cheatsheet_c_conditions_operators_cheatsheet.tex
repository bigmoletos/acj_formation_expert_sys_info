\subsection*{Conditions en C}

Les instructions conditionnelles permettent de contrôler le flux d\textquotesingle{}exécution d\textquotesingle{}un programme en fonction de certaines conditions.

\subsubsection*{Syntaxe de base \+:}

\paragraph*{1. {\ttfamily if}}


\begin{DoxyCode}
\textcolor{keywordflow}{if} (condition) \{
    \textcolor{comment}{// code exécuté si la condition est vraie}
\}
\end{DoxyCode}


\paragraph*{2. {\ttfamily if-\/else}}


\begin{DoxyCode}
\textcolor{keywordflow}{if} (condition) \{
    \textcolor{comment}{// code exécuté si la condition est vraie}
\} \textcolor{keywordflow}{else} \{
    \textcolor{comment}{// code exécuté si la condition est fausse}
\}
\end{DoxyCode}


\paragraph*{3. {\ttfamily else if}}


\begin{DoxyCode}
\textcolor{keywordflow}{if} (condition1) \{
    \textcolor{comment}{// code exécuté si condition1 est vraie}
\} \textcolor{keywordflow}{else} \textcolor{keywordflow}{if} (condition2) \{
    \textcolor{comment}{// code exécuté si condition2 est vraie}
\} \textcolor{keywordflow}{else} \{
    \textcolor{comment}{// code exécuté si aucune condition n'est vraie}
\}
\end{DoxyCode}


\paragraph*{4. {\ttfamily switch-\/case}}


\begin{DoxyCode}
\textcolor{keywordflow}{switch} (variable) \{
    \textcolor{keywordflow}{case} valeur1:
        \textcolor{comment}{// code si variable == valeur1}
        \textcolor{keywordflow}{break};
    \textcolor{keywordflow}{case} valeur2:
        \textcolor{comment}{// code si variable == valeur2}
        \textcolor{keywordflow}{break};
    \textcolor{keywordflow}{default}:
        \textcolor{comment}{// code si aucune des valeurs n'est trouvée}
        \textcolor{keywordflow}{break};
\}
\end{DoxyCode}


\subsection*{Opérateurs en C}

\subsubsection*{1. Opérateurs arithmétiques}


\begin{DoxyItemize}
\item {\ttfamily +} \+: Addition
\item {\ttfamily -\/} \+: Soustraction
\item {\ttfamily $\ast$} \+: Multiplication
\item {\ttfamily /} \+: Division
\item {\ttfamily \%} \+: Modulo (reste de la division entière)
\end{DoxyItemize}

Exemple \+: 
\begin{DoxyCode}
\textcolor{keywordtype}{int} a = 5, b = 2;
\textcolor{keywordtype}{int} somme = a + b;     \textcolor{comment}{// somme = 7}
\textcolor{keywordtype}{int} produit = a * b;   \textcolor{comment}{// produit = 10}
\textcolor{keywordtype}{int} division = a / b;  \textcolor{comment}{// division = 2 (division entière)}
\textcolor{keywordtype}{int} reste = a % b;     \textcolor{comment}{// reste = 1}
\end{DoxyCode}


\subsubsection*{2. Opérateurs de comparaison}


\begin{DoxyItemize}
\item {\ttfamily ==} \+: Égal à
\item {\ttfamily !=} \+: Différent de
\item {\ttfamily $>$} \+: Supérieur à
\item {\ttfamily $<$} \+: Inférieur à
\item {\ttfamily $>$=} \+: Supérieur ou égal à
\item {\ttfamily $<$=} \+: Inférieur ou égal à
\end{DoxyItemize}

Exemple \+: 
\begin{DoxyCode}
\textcolor{keywordflow}{if} (a == b) \{
    \textcolor{comment}{// code exécuté si a est égal à b}
\}
\end{DoxyCode}


\subsubsection*{3. Opérateurs logiques}


\begin{DoxyItemize}
\item {\ttfamily \&\&} \+: ET logique (vrai si les deux conditions sont vraies)
\item {\ttfamily $\vert$$\vert$} \+: OU logique (vrai si au moins une condition est vraie)
\item {\ttfamily !} \+: N\+ON logique (inverse de la condition)
\end{DoxyItemize}

Exemple \+: 
\begin{DoxyCode}
\textcolor{keywordflow}{if} (a > 0 && b < 10) \{
    \textcolor{comment}{// code exécuté si a est supérieur à 0 ET b est inférieur à 10}
\}
\end{DoxyCode}


\subsubsection*{4. Opérateurs d\textquotesingle{}incrémentation et de décrémentation}


\begin{DoxyItemize}
\item {\ttfamily ++} \+: Incrémentation (ajoute 1 à la valeur)
\item {\ttfamily -\/-\/} \+: Décrémentation (soustrait 1 à la valeur)
\end{DoxyItemize}

Exemple \+: 
\begin{DoxyCode}
\textcolor{keywordtype}{int} x = 5;
x++;  \textcolor{comment}{// x vaut maintenant 6}
x--;  \textcolor{comment}{// x vaut maintenant 5}
\end{DoxyCode}


\subsubsection*{5. Opérateurs d\textquotesingle{}affectation}


\begin{DoxyItemize}
\item {\ttfamily =} \+: Affectation
\item {\ttfamily +=} \+: Ajout puis affectation
\item {\ttfamily -\/=} \+: Soustraction puis affectation
\item {\ttfamily $\ast$=} \+: Multiplication puis affectation
\item {\ttfamily /=} \+: Division puis affectation
\item {\ttfamily \%=} \+: Modulo puis affectation
\end{DoxyItemize}

Exemple \+: 
\begin{DoxyCode}
\textcolor{keywordtype}{int} x = 10;
x += 5;   \textcolor{comment}{// x vaut maintenant 15}
x *= 2;   \textcolor{comment}{// x vaut maintenant 30}
\end{DoxyCode}


\subsubsection*{6. Opérateurs bit à bit (bitwise)}


\begin{DoxyItemize}
\item {\ttfamily \&} \+: ET bit à bit
\item {\ttfamily $\vert$} \+: OU bit à bit
\item {\ttfamily $^\wedge$} \+: X\+OR (OU exclusif) bit à bit
\item {\ttfamily $\sim$} \+: N\+ON bit à bit
\item {\ttfamily $<$$<$} \+: Décalage à gauche
\item {\ttfamily $>$$>$} \+: Décalage à droite
\end{DoxyItemize}

Exemple \+: 
\begin{DoxyCode}
\textcolor{keywordtype}{int} x = 5;  \textcolor{comment}{// binaire : 0101}
\textcolor{keywordtype}{int} y = 3;  \textcolor{comment}{// binaire : 0011}

\textcolor{keywordtype}{int} z = x & \hyperlink{addition_8c_a0a2f84ed7838f07779ae24c5a9086d33}{y};  \textcolor{comment}{// z = 1  (binaire : 0001)}
\textcolor{keywordtype}{int} w = x | \hyperlink{addition_8c_a0a2f84ed7838f07779ae24c5a9086d33}{y};  \textcolor{comment}{// w = 7  (binaire : 0111)}
\end{DoxyCode}


\subsubsection*{7. Opérateurs ternaires}

L\textquotesingle{}opérateur ternaire est une forme raccourcie de la structure {\ttfamily if-\/else}.

Syntaxe \+: 
\begin{DoxyCode}
condition ? expression1 : expression2;
\end{DoxyCode}


Exemple \+: 
\begin{DoxyCode}
\textcolor{keywordtype}{int} a = 10, b = 20;
\textcolor{keywordtype}{int} max = (a > b) ? a : b;  \textcolor{comment}{// max vaut 20}
\end{DoxyCode}
 