Ce cheatsheet couvre les principales structures de données en C++ et propose des exemples pratiques pour les utiliser \+: {\ttfamily std\+::list}, {\ttfamily std\+::vector}, {\ttfamily std\+::set}, {\ttfamily std\+::map}, {\ttfamily std\+::stack}, {\ttfamily std\+::queue}, {\ttfamily std\+::multiset}, et {\ttfamily std\+::multimap}. 



\subsection*{1. std\+::list}

\subsubsection*{Description}

Liste doublement chaînée. Permet des insertions et suppressions efficaces.

\#\#\# Exemples \+: 
\begin{DoxyCode}
\textcolor{preprocessor}{#include <list>}
\textcolor{preprocessor}{#include <iostream>}
\textcolor{keyword}{using namespace }\hyperlink{namespacestd}{std};

\textcolor{keywordtype}{int} \hyperlink{htop_8c_a3c04138a5bfe5d72780bb7e82a18e627}{main}() \{
    list<int> myList = \{1, 2, 3\};
    myList.push\_back(4); \textcolor{comment}{// Ajouter à la fin}
    myList.push\_front(0); \textcolor{comment}{// Ajouter au début}

    myList.remove(2); \textcolor{comment}{// Supprimer la valeur 2}

    \textcolor{keywordflow}{for} (\textcolor{keywordtype}{int} x : myList) \{
        cout << x << \textcolor{stringliteral}{" "};
    \}
    \textcolor{keywordflow}{return} 0;
\}
\end{DoxyCode}




\subsection*{2. std\+::vector}

\subsubsection*{Description}

Tableau dynamique avec accès rapide aux éléments par index.

\#\#\# Exemples \+: 
\begin{DoxyCode}
\textcolor{preprocessor}{#include <vector>}
\textcolor{preprocessor}{#include <iostream>}
\textcolor{keyword}{using namespace }\hyperlink{namespacestd}{std};

\textcolor{keywordtype}{int} \hyperlink{htop_8c_a3c04138a5bfe5d72780bb7e82a18e627}{main}() \{
    vector<int> myVector = \{1, 2, 3\};
    myVector.push\_back(4); \textcolor{comment}{// Ajouter à la fin}

    myVector[1] = 5; \textcolor{comment}{// Modifier l'élément à l'index 1}

    myVector.pop\_back(); \textcolor{comment}{// Supprimer le dernier élément}

    \textcolor{keywordflow}{for} (\textcolor{keywordtype}{int} x : myVector) \{
        cout << x << \textcolor{stringliteral}{" "};
    \}
    \textcolor{keywordflow}{return} 0;
\}
\end{DoxyCode}




\subsection*{3. std\+::set}

\subsubsection*{Description}

Conteneur ordonné qui ne contient que des éléments uniques.

\#\#\# Exemples \+: 
\begin{DoxyCode}
\textcolor{preprocessor}{#include <set>}
\textcolor{preprocessor}{#include <iostream>}
\textcolor{keyword}{using namespace }\hyperlink{namespacestd}{std};

\textcolor{keywordtype}{int} \hyperlink{htop_8c_a3c04138a5bfe5d72780bb7e82a18e627}{main}() \{
    set<int> mySet = \{1, 2, 3\};
    mySet.insert(4); \textcolor{comment}{// Insérer un élément}
    mySet.erase(2); \textcolor{comment}{// Supprimer un élément}

    \textcolor{keywordflow}{for} (\textcolor{keywordtype}{int} x : mySet) \{
        cout << x << \textcolor{stringliteral}{" "};
    \}
    \textcolor{keywordflow}{return} 0;
\}
\end{DoxyCode}




\subsection*{4. std\+::map}

\subsubsection*{Description}

Table associative clé-\/valeur.

\#\#\# Exemples \+: 
\begin{DoxyCode}
\textcolor{preprocessor}{#include <map>}
\textcolor{preprocessor}{#include <iostream>}
\textcolor{keyword}{using namespace }\hyperlink{namespacestd}{std};

\textcolor{keywordtype}{int} \hyperlink{htop_8c_a3c04138a5bfe5d72780bb7e82a18e627}{main}() \{
    map<string, int> myMap;
    myMap[\textcolor{stringliteral}{"apple"}] = 5; \textcolor{comment}{// Ajouter ou modifier}
    myMap[\textcolor{stringliteral}{"banana"}] = 2;

    myMap.erase(\textcolor{stringliteral}{"apple"}); \textcolor{comment}{// Supprimer une clé}

    \textcolor{keywordflow}{for} (\textcolor{keyword}{auto} &[\hyperlink{Action_8c_acd3d88da3c0e0313c3645ff34f62f542}{key}, value] : myMap) \{
        cout << \hyperlink{Action_8c_acd3d88da3c0e0313c3645ff34f62f542}{key} << \textcolor{stringliteral}{": "} << value << endl;
    \}
    \textcolor{keywordflow}{return} 0;
\}
\end{DoxyCode}




\subsection*{5. std\+::stack}

\subsubsection*{Description}

Pile (L\+I\+FO \+: dernier entré, premier sorti).

\#\#\# Exemples \+: 
\begin{DoxyCode}
\textcolor{preprocessor}{#include <stack>}
\textcolor{preprocessor}{#include <iostream>}
\textcolor{keyword}{using namespace }\hyperlink{namespacestd}{std};

\textcolor{keywordtype}{int} \hyperlink{htop_8c_a3c04138a5bfe5d72780bb7e82a18e627}{main}() \{
    stack<int> myStack;
    myStack.push(1); \textcolor{comment}{// Ajouter un élément}
    myStack.push(2);

    myStack.pop(); \textcolor{comment}{// Retirer le dernier élément}

    cout << \textcolor{stringliteral}{"Top element: "} << myStack.top() << endl;
    \textcolor{keywordflow}{return} 0;
\}
\end{DoxyCode}




\subsection*{6. std\+::queue}

\subsubsection*{Description}

File (F\+I\+FO \+: premier entré, premier sorti).

\#\#\# Exemples \+: 
\begin{DoxyCode}
\textcolor{preprocessor}{#include <queue>}
\textcolor{preprocessor}{#include <iostream>}
\textcolor{keyword}{using namespace }\hyperlink{namespacestd}{std};

\textcolor{keywordtype}{int} \hyperlink{htop_8c_a3c04138a5bfe5d72780bb7e82a18e627}{main}() \{
    queue<int> myQueue;
    myQueue.push(1); \textcolor{comment}{// Ajouter un élément}
    myQueue.push(2);

    myQueue.pop(); \textcolor{comment}{// Retirer le premier élément}

    cout << \textcolor{stringliteral}{"Front element: "} << myQueue.front() << endl;
    \textcolor{keywordflow}{return} 0;
\}
\end{DoxyCode}




\subsection*{7. std\+::multiset}

\subsubsection*{Description}

Conteneur ordonné qui permet des éléments dupliqués.

\#\#\# Exemples \+: 
\begin{DoxyCode}
\textcolor{preprocessor}{#include <set>}
\textcolor{preprocessor}{#include <iostream>}
\textcolor{keyword}{using namespace }\hyperlink{namespacestd}{std};

\textcolor{keywordtype}{int} \hyperlink{htop_8c_a3c04138a5bfe5d72780bb7e82a18e627}{main}() \{
    multiset<int> myMultiset = \{1, 2, 2, 3\};
    myMultiset.insert(4); \textcolor{comment}{// Ajouter un élément}
    myMultiset.erase(myMultiset.find(2)); \textcolor{comment}{// Supprimer un élément (une seule occurrence)}

    \textcolor{keywordflow}{for} (\textcolor{keywordtype}{int} x : myMultiset) \{
        cout << x << \textcolor{stringliteral}{" "};
    \}
    \textcolor{keywordflow}{return} 0;
\}
\end{DoxyCode}




\subsection*{8. std\+::multimap}

\subsubsection*{Description}

Table associative clé-\/valeur qui permet des clés dupliquées.

\#\#\# Exemples \+: 
\begin{DoxyCode}
\textcolor{preprocessor}{#include <map>}
\textcolor{preprocessor}{#include <iostream>}
\textcolor{keyword}{using namespace }\hyperlink{namespacestd}{std};

\textcolor{keywordtype}{int} \hyperlink{htop_8c_a3c04138a5bfe5d72780bb7e82a18e627}{main}() \{
    multimap<string, int> myMultimap;
    myMultimap.insert(\{\textcolor{stringliteral}{"apple"}, 5\});
    myMultimap.insert(\{\textcolor{stringliteral}{"apple"}, 3\}); \textcolor{comment}{// Clé dupliquée}

    \textcolor{keywordflow}{for} (\textcolor{keyword}{auto} &[\hyperlink{Action_8c_acd3d88da3c0e0313c3645ff34f62f542}{key}, value] : myMultimap) \{
        cout << \hyperlink{Action_8c_acd3d88da3c0e0313c3645ff34f62f542}{key} << \textcolor{stringliteral}{": "} << value << endl;
    \}
    \textcolor{keywordflow}{return} 0;
\}
\end{DoxyCode}




\subsubsection*{Notes Générales}


\begin{DoxyItemize}
\item Inclure les bons headers pour chaque conteneur.
\item Adapter les structures de données aux besoins spécifiques en performance et contraintes. 
\end{DoxyItemize}