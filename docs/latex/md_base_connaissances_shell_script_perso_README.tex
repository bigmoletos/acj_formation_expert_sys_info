Ce répertoire contient des scripts automatisés pour l\textquotesingle{}installation de Kubernetes sur des machines Ubuntu, avec une configuration master-\/worker.

\subsection*{Structure du Projet}


\begin{DoxyCode}
C:\(\backslash\)AJC\_formation\(\backslash\)docker\(\backslash\)kubernetes\(\backslash\)script\_perso\(\backslash\)
├── README.md
├── install\_k8s\_master.sh
└── install\_k8s\_worker.sh
\end{DoxyCode}


\subsection*{Prérequis}


\begin{DoxyItemize}
\item Ubuntu Server (recommandé \+: 20.\+04 L\+TS ou plus récent)
\item Minimum 2 GB R\+AM par machine
\item Minimum 2 C\+PU par machine
\item Connexion réseau entre les machines
\item Droits sudo
\item Ports ouverts \+:
\begin{DoxyItemize}
\item Master \+: 6443, 2379-\/2380, 10250, 10251, 10252
\item Worker \+: 10250, 30000-\/32767
\end{DoxyItemize}
\end{DoxyItemize}

\subsection*{Installation}


\begin{DoxyEnumerate}
\item Cloner ou copier les scripts sur les machines respectives \+: 
\begin{DoxyCode}
git clone <url\_repo> /tmp/k8s-scripts
cd /tmp/k8s-scripts
chmod +x install\_k8s\_*.sh
\end{DoxyCode}

\item Sur le nœud master \+: 
\begin{DoxyCode}
sudo ./install\_k8s\_master.sh
# ou en mode automatique
sudo ./install\_k8s\_master.sh -y
\end{DoxyCode}

\item Noter les informations de jointure fournies à la fin de l\textquotesingle{}installation du master
\item Sur chaque nœud worker \+: 
\begin{DoxyCode}
sudo ./install\_k8s\_worker.sh <numero\_worker>
# ou en mode automatique avec les informations du master
sudo ./install\_k8s\_worker.sh -y <numero\_worker> <ip\_master> <token> <hash>
\end{DoxyCode}

\end{DoxyEnumerate}

\subsection*{Options des Scripts}

\#\#\# Script Master (install\+\_\+k8s\+\_\+master.\+sh) 
\begin{DoxyCode}
Options:
    -h, --help      Affiche l'aide
    -y, --yes       Mode automatique (pas de demande de confirmation)
\end{DoxyCode}


\#\#\# Script Worker (install\+\_\+k8s\+\_\+worker.\+sh) 
\begin{DoxyCode}
Options:
    -h, --help      Affiche l'aide
    -y, --yes       Mode automatique
Arguments:
    worker\_number   Numéro du worker (obligatoire)
    master\_ip       IP du master (optionnel en mode interactif)
    token          Token de jointure (optionnel en mode interactif)
    discovery\_hash  Hash de découverte (optionnel en mode interactif)
\end{DoxyCode}


\subsection*{Logs et Résumé}

Les scripts génèrent deux fichiers \+:
\begin{DoxyItemize}
\item {\ttfamily k8s\+\_\+install.\+log} \+: Journal détaillé de l\textquotesingle{}installation
\item {\ttfamily k8s\+\_\+summary.\+md} \+: Résumé des actions effectuées
\end{DoxyItemize}

\subsection*{Vérification de l\textquotesingle{}Installation}

Sur le nœud master, vérifiez l\textquotesingle{}état du cluster \+: 
\begin{DoxyCode}
kubectl get nodes
kubectl get pods --all-namespaces
\end{DoxyCode}


\subsection*{Dépannage}

Si l\textquotesingle{}installation échoue \+:


\begin{DoxyEnumerate}
\item Vérifiez les logs \+: 
\begin{DoxyCode}
cat k8s\_install.log
\end{DoxyCode}

\item Reset Kubernetes si nécessaire \+: 
\begin{DoxyCode}
sudo kubeadm reset
sudo systemctl stop kubelet
sudo systemctl disable kubelet
sudo rm -rf /etc/kubernetes/
\end{DoxyCode}

\item Vérifiez les ports \+: 
\begin{DoxyCode}
sudo netstat -tulpn | grep -E '6443|2379|2380|10250|10251|10252'
\end{DoxyCode}

\end{DoxyEnumerate}

\subsection*{Documentation Supplémentaire}

Pour plus d\textquotesingle{}informations, consultez \+:
\begin{DoxyItemize}
\item \href{https://kubernetes.io/docs/home/}{\tt Documentation Kubernetes}
\item \href{https://kubernetes.io/docs/setup/}{\tt Guide d\textquotesingle{}installation Kubernetes}
\item Les fichiers dans {\ttfamily C\+:\textbackslash{}A\+J\+C\+\_\+formation\textbackslash{}docker\textbackslash{}kubernetes\textbackslash{}} pour des informations spécifiques à votre environnement
\end{DoxyItemize}

\subsection*{Contribution}

Pour contribuer à l\textquotesingle{}amélioration de ces scripts \+:
\begin{DoxyEnumerate}
\item Forkez le projet
\item Créez une branche pour votre fonctionnalité
\item Soumettez une pull request
\end{DoxyEnumerate}

\subsection*{Licence}

Ces scripts sont fournis \char`\"{}tels quels\char`\"{}, sans garantie d\textquotesingle{}aucune sorte.

\subsection*{Auteur}

\mbox{[}Votre nom\mbox{]}

\subsection*{Dernière mise à jour}

\mbox{[}Date de la dernière mise à jour\mbox{]} 