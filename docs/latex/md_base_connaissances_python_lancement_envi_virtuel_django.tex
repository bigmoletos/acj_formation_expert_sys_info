python.\+exe -\/m pip install --upgrade pip

\section*{Installer virtualenv}

pip install virtualenv

\section*{Créer un environnement virtuel}

virtualenv django\+\_\+env

\section*{Activer l\textquotesingle{}environnement virtuel}

.

\section*{Installer les packages}

pip install django views pandas numpy plotly seaborn scikit-\/learn scipy bs4 fuzzywuzzy django-\/select2 requests joblib djangorestframework selenium webdriver\+\_\+manager

\section*{Enregistrer les packages installés dans un fichier requirements.\+txt}

pip freeze $>$ requirements.\+txt

\section*{Installer les packages à partir du fichier requirements.\+txt en adaptant les verssion à notre environnement}

pip install -\/r requirements.\+txt

\section*{Créer un projet Django}

django-\/admin startproject projet\+\_\+recommandation\+\_\+films

\section*{lancer le fichier app.\+py}

python manage.\+py runserver

\section*{Mise à jour des migrations de B\+DD}

python manage.\+py migrate

\section*{relancer le fichier app.\+py}

python manage.\+py runserver

\section*{\href{http://127.0.0.1:5000}{\tt http\+://127.\+0.\+0.\+1\+:5000} sur le port 5000. Donc, l’\+U\+RL serait \href{http://localhost:5000}{\tt http\+://localhost\+:5000} ou \href{http://127.0.0.1:5000}{\tt http\+://127.\+0.\+0.\+1\+:5000}.}

\section*{pour arreter l\textquotesingle{}environnement virtuel}

deactivate

\section*{Pour relancer le projet}

\subsection*{se mettre dans le dossier $\ast$$\ast$$\ast$\+Projet2\+\_\+\+Net\+Flix$>$$\ast$$\ast$$\ast$}

cd Django . \subsection*{se mettre dans le dossier $\ast$$\ast$$\ast$\+Projet2\+\_\+\+Net\+Flix$>$$\ast$$\ast$$\ast$}

cd .\textbackslash{} python manage.\+py runserver

\section*{importer réguliérement les données statiques du site css, js ,image etc....}

python manage.\+py collectstatic

\section*{si besoin de désinstaller des prog}

pip uninstall django pandas numpy plotly seaborn scikit-\/learn

\section*{les templates par défaut doivent se trouver dans le repertoire à la racine de app.\+py dans un dossier templates}

créer un dossier {\itshape {\bfseries my\+\_\+application}} créer un dossier {\itshape {\bfseries template}} comprenant tous les fichiers html Django\textbackslash{} accueil.\+html \subsubsection*{les fichiers importants se trouvent dans\+:}

Projets\textbackslash{} manage.\+py Projets\textbackslash{} models.\+py Django\textbackslash{} views.\+py \subsubsection*{Django$<$/h3$>$ Django\textbackslash{} urls.\+py settings.\+py les fichiers python de l\textquotesingle{}application}

Django\textbackslash{} stat\+\_\+acteur.\+py

\section*{Créez une application en utilisant manage.\+py. Assurez-\/vous d\textquotesingle{}être dans le même répertoire que manage.\+py et exécutez la commande suivante}

\subsubsection*{on ne doit pas avoir plusieurs manage.\+py dans le projet}

python manage.\+py startapp nom\+\_\+de\+\_\+votre\+\_\+application

\section*{Gestion des problemes on peut forcer lma réinstallation de django pour qu\textquotesingle{}il prenne en compte notre architecture}

pip install --force-\/reinstall django

\section*{pour annuler un e meigration}

python manage.\+py migrate anom\+\_\+application zero \section*{si aprés des pb de migration il faut recreer les dossiers migration d\textquotesingle{}un application}

python manage.\+py makemigrations nom\+\_\+application python manage.\+py migrate

\section*{pour créer un user admin}

python manage.\+py createsuperuser

\section*{Changer le mot de passe d\textquotesingle{}un superutilisateur existant}

python manage.\+py changepassword $<$username$>$ 