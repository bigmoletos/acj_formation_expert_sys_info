\subsection*{Objectif}

Ce document explique comment compiler et exécuter un programme C sous W\+SL en utilisant le compilateur {\bfseries G\+CC}.

\subsection*{Prérequis}


\begin{DoxyItemize}
\item Avoir installé {\bfseries W\+SL} (Windows Subsystem for Linux) sur votre machine.
\item Avoir installé {\bfseries G\+CC} dans l\textquotesingle{}environnement W\+SL.
\item pour {\bfseries connaitre la version} de son compilateur et de son C
\end{DoxyItemize}


\begin{DoxyCode}
gcc --version
gcc -dM -E - < /dev/null | grep \_\_STDC\_VERSION\_\_
\end{DoxyCode}


\subsubsection*{Installation de G\+CC sous W\+SL}

Si G\+CC n\textquotesingle{}est pas installé, voici les étapes pour l\textquotesingle{}installer \+:


\begin{DoxyEnumerate}
\item Mettre à jour la liste des paquets \+:
\end{DoxyEnumerate}


\begin{DoxyCode}
sudo apt update
sudo apt upgrade
sudo apt install build-essential -y
sudo apt install mingw-w64
\end{DoxyCode}



\begin{DoxyEnumerate}
\item Installer le compilateur G\+CC \+: 
\begin{DoxyCode}
sudo apt install gcc
\end{DoxyCode}

\item Vérifier l\textquotesingle{}installation de G\+CC \+: 
\begin{DoxyCode}
gcc --version
\end{DoxyCode}

\end{DoxyEnumerate}

\subsection*{Étapes pour compiler un programme C}

\subsubsection*{1. Créer un fichier source C}


\begin{DoxyEnumerate}
\item Ouvrir un éditeur de texte sous W\+SL (par exemple, {\ttfamily nano} ou {\ttfamily vim}). 
\begin{DoxyCode}
nano hello.c
\end{DoxyCode}

\item Écrire le programme C dans ce fichier \+:
\end{DoxyEnumerate}


\begin{DoxyCode}
#include <stdio.h>

int main() \{
      printf("Hello, World!\(\backslash\)n");
      return 0;
\}
\end{DoxyCode}



\begin{DoxyEnumerate}
\item Sauvegarder et quitter l\textquotesingle{}éditeur \+:
\begin{DoxyItemize}
\item Pour {\ttfamily nano} \+: {\ttfamily C\+T\+R\+L+O} pour enregistrer, puis {\ttfamily C\+T\+R\+L+X} pour quitter.
\end{DoxyItemize}
\end{DoxyEnumerate}

\subsubsection*{2. Compiler le fichier source}


\begin{DoxyEnumerate}
\item Utiliser G\+CC pour compiler le fichier source \+:
\end{DoxyEnumerate}


\begin{DoxyCode}
gcc hello.c -o hello
\end{DoxyCode}



\begin{DoxyEnumerate}
\item Cette commande crée un fichier exécutable appelé {\ttfamily hello}.
\item Compiler un exe fonctionnant sous windows \+:
\end{DoxyEnumerate}


\begin{DoxyCode}
export LANG=C.UTF-8 # pour assurer que l'encodage du fichier sera en utf8
x86\_64-w64-mingw32-gcc -o mon\_prog.exe fichier\_source.c
\end{DoxyCode}



\begin{DoxyEnumerate}
\item Cette commande crée un fichier exécutable appelé {\ttfamily hello}. \subsubsection*{3. Exécuter le programme compilé}
\end{DoxyEnumerate}


\begin{DoxyEnumerate}
\item Exécuter l\textquotesingle{}exécutable généré avec la commande suivante \+: 
\begin{DoxyCode}
./hello
\end{DoxyCode}

\item Vous devriez voir le message suivant s\textquotesingle{}afficher \+: 
\begin{DoxyCode}
Hello, World!
\end{DoxyCode}

\end{DoxyEnumerate}

\subsection*{Options de compilation courantes}


\begin{DoxyItemize}
\item {\bfseries -\/o} \+: Spécifie le nom de l\textquotesingle{}exécutable. Exemple \+: 
\begin{DoxyCode}
gcc fichier.c -o nom\_programme
\end{DoxyCode}

\item {\bfseries -\/\+Wall} \+: Active tous les avertissements. Exemple \+: 
\begin{DoxyCode}
gcc -Wall fichier.c -o programme
\end{DoxyCode}

\item {\bfseries -\/g} \+: Inclut des informations de débogage dans l\textquotesingle{}exécutable. Exemple \+: 
\begin{DoxyCode}
gcc -g fichier.c -o programme
\end{DoxyCode}

\item {\bfseries -\/\+O2} \+: Active des optimisations de niveau 2 pour améliorer les performances. Exemple \+: 
\begin{DoxyCode}
gcc -O2 fichier.c -o programme
\end{DoxyCode}

\item {\bfseries -\/lm} \+: Lien avec la bibliothèque mathématique {\ttfamily libm} pour utiliser des fonctions mathématiques. Exemple \+: 
\begin{DoxyCode}
gcc fichier.c -o programme -lm
\end{DoxyCode}

\end{DoxyItemize}

\subsection*{Variables}

```bash \#include $<$stdio.\+h$>$

int \hyperlink{htop_8c_a3c04138a5bfe5d72780bb7e82a18e627}{main()} \{ // Déclarations de variables de différents types int int\+\_\+var = 42; long long\+\_\+var = 1234567890L; unsigned int uint\+\_\+var = 300; unsigned long ulong\+\_\+var = 4000000000\+UL; float float\+\_\+var = 3.\+14159f; double double\+\_\+var = 2.\+718281828459; long double ldouble\+\_\+var = 1.\+61803398875L; char char\+\_\+var = \textquotesingle{}A\textquotesingle{}; char string\+\_\+var\mbox{[}\mbox{]} = \char`\"{}\+Hello, World!\char`\"{};

// Impression des différentes variables avec les formats appropriés printf(\char`\"{}\+Int\+: \%d\textbackslash{}n\char`\"{}, int\+\_\+var); printf(\char`\"{}\+Long Int\+: \%ld\textbackslash{}n\char`\"{}, long\+\_\+var); printf(\char`\"{}\+Unsigned Int\+: \%u\textbackslash{}n\char`\"{}, uint\+\_\+var); printf(\char`\"{}\+Unsigned Long Int\+: \%lu\textbackslash{}n\char`\"{}, ulong\+\_\+var);

// Représentation en octal et hexadécimal printf(\char`\"{}\+Unsigned Int (octal)\+: \%o\textbackslash{}n\char`\"{}, uint\+\_\+var); printf(\char`\"{}\+Unsigned Int (hexadécimal)\+: \%x\textbackslash{}n\char`\"{}, uint\+\_\+var); printf(\char`\"{}\+Unsigned Long Int (hexadécimal)\+: \%lx\textbackslash{}n\char`\"{}, ulong\+\_\+var);

// Valeurs flottantes et doubles printf(\char`\"{}\+Float\+: \%f\textbackslash{}n\char`\"{}, float\+\_\+var); printf(\char`\"{}\+Double\+: \%f\textbackslash{}n\char`\"{}, double\+\_\+var); printf(\char`\"{}\+Long Double\+: \%\+Lf\textbackslash{}n\char`\"{}, ldouble\+\_\+var);

// Notation exponentielle printf(\char`\"{}\+Double (exponentielle)\+: \%e\textbackslash{}n\char`\"{}, double\+\_\+var); printf(\char`\"{}\+Long Double (exponentielle)\+: \%\+Le\textbackslash{}n\char`\"{}, ldouble\+\_\+var);

// Notation courte entre f et e printf(\char`\"{}\+Double (notation courte)\+: \%g\textbackslash{}n\char`\"{}, double\+\_\+var); printf(\char`\"{}\+Long Double (notation courte)\+: \%\+Lg\textbackslash{}n\char`\"{}, ldouble\+\_\+var);

// Caractère et chaîne de caractères printf(\char`\"{}\+Char\+: \%c\textbackslash{}n\char`\"{}, char\+\_\+var); printf(\char`\"{}\+String\+: \%s\textbackslash{}n\char`\"{}, string\+\_\+var);

return 0; \} 

 