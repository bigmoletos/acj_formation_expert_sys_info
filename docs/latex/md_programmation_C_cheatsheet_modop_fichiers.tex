\subsection*{Ouverture et Fermeture de Fichier}

\#\#\# {\ttfamily fopen} \+: Ouvrir un fichier 
\begin{DoxyCode}
FILE *fopen(const char *nomFichier, const char *mode);
nomFichier : Nom du fichier à ouvrir.
mode : Mode d'ouverture du fichier :
"r" : Lecture
"w" : Écriture (crée un fichier s'il n'existe pas)
"a" : Ajout (crée un fichier s'il n'existe pas)
"r+" : Lecture/Écriture
"w+" : Écriture/Lecture
"a+" : Ajout/Lecture
\end{DoxyCode}



\begin{DoxyItemize}
\item Exemple \+:
\end{DoxyItemize}


\begin{DoxyCode}
FILE *fp = fopen("mon\_fichier.txt", "r");
fclose : Fermer un fichier

int fclose(FILE *fp);

//Ferme le fichier ouvert.
//Renvoie 0 en cas de succès.
\end{DoxyCode}

\begin{DoxyItemize}
\item Exemple \+:
\end{DoxyItemize}


\begin{DoxyCode}
fclose(fp);
Lecture de Fichiers
fscanf : Lire du contenu formaté

int fscanf(FILE *fp, const char *format, ...);
Lit des données formatées à partir d'un fichier.
\end{DoxyCode}



\begin{DoxyItemize}
\item Exemple \+:
\end{DoxyItemize}


\begin{DoxyCode}
int age;
fscanf(fp, "%d", &age);
fgets : Lire une ligne

char *fgets(char *str, int n, FILE *fp);
Lit une chaîne de caractères de longueur maximale n à partir du fichier fp.
\end{DoxyCode}



\begin{DoxyItemize}
\item Exemple \+:
\end{DoxyItemize}


\begin{DoxyCode}
char ligne[100];
fgets(ligne, 100, fp);
Écriture dans des Fichiers
fprintf : Écrire du contenu formaté

int fprintf(FILE *fp, const char *format, ...);
Écrit des données formatées dans un fichier.
\end{DoxyCode}

\begin{DoxyItemize}
\item Exemple \+:
\end{DoxyItemize}


\begin{DoxyCode}
fprintf(fp, "Nom: %s, Âge: %d\(\backslash\)n", nom, age);
fputs : Écrire une chaîne de caractères

int fputs(const char *str, FILE *fp);
Écrit une chaîne de caractères dans un fichier.
\end{DoxyCode}


Exemple \+:


\begin{DoxyCode}
fputs("Bonjour, monde !", fp);
Autres Fonctions Utiles
fseek : Déplacer le pointeur de fichier

int fseek(FILE *fp, long offset, int origine);
offset : Déplacement en octets.
origine :
SEEK\_SET : Depuis le début.
SEEK\_CUR : Depuis la position actuelle.
SEEK\_END : Depuis la fin.
Exemple :


fseek(fp, 0, SEEK\_SET);  // Retourner au début du fichier
ftell : Obtenir la position actuelle du pointeur de fichier

long ftell(FILE *fp);
Renvoie la position actuelle dans le fichier.
\end{DoxyCode}



\begin{DoxyItemize}
\item Exemple
\end{DoxyItemize}


\begin{DoxyCode}
long pos = ftell(fp);
rewind : Remettre le pointeur au début du fichier

void rewind(FILE *fp);
\end{DoxyCode}

\begin{DoxyItemize}
\item Exemple
\end{DoxyItemize}


\begin{DoxyCode}
rewind(fp);
feof : Vérifier la fin d'un fichier

int feof(FILE *fp);
Renvoie une valeur non nulle si la fin du fichier est atteinte.
\end{DoxyCode}

\begin{DoxyItemize}
\item Exemple
\end{DoxyItemize}


\begin{DoxyCode}
if (feof(fp)) \{
    printf("Fin du fichier atteinte.\(\backslash\)n");
\}
ferror : Vérifier les erreurs

int ferror(FILE *fp);
Renvoie une valeur non nulle si une erreur s'est produite.
\end{DoxyCode}

\begin{DoxyItemize}
\item Exemple
\end{DoxyItemize}


\begin{DoxyCode}
if (ferror(fp)) \{
    printf("Erreur lors de la lecture du fichier.\(\backslash\)n");
\}
\end{DoxyCode}
 \subsection*{Exemple Complet d\textquotesingle{}Utilisation Lecture d\textquotesingle{}un fichier ligne par ligne}


\begin{DoxyCode}
FILE *fp = fopen("mon\_fichier.txt", "r");
if (fp == NULL) \{
    perror("Erreur d'ouverture de fichier");
    return 1;
\}

char ligne[100];
while (fgets(ligne, sizeof(ligne), fp)) \{
    printf("%s", ligne);
\}

fclose(fp);
Écriture dans un fichier

FILE *fp = fopen("sortie.txt", "w");
if (fp == NULL) \{
    perror("Erreur d'ouverture de fichier");
    return 1;
\}

fprintf(fp, "Ceci est un exemple de texte.\(\backslash\)n");

fclose(fp);
\end{DoxyCode}
 