
\begin{DoxyCode}
\{python.exe\}

#  installer Streamlit
```pip install streamli
\end{DoxyCode}


\# Installer virtualenv 
\begin{DoxyCode}
# Premier script Streamlit
## Créez un fichier, par exemple app.py, puis ajoutez
```bash
import streamlit as st
st.title('Mon premier app Streamlit')
st.write('Bienvenue dans cette application.')
\end{DoxyCode}


\# Créer un environnement virtuel 
\begin{DoxyCode}
# Activer l'environnement virtuel
```.\(\backslash\)streamlit\_env\(\backslash\)Scripts\(\backslash\)activat
\end{DoxyCode}


\# Installer les packages 
\begin{DoxyCode}
# Enregistrer les packages installés dans un fichier requirements.txt
```pip freeze > requirements.tx
\end{DoxyCode}


\# Installer les packages à partir du fichier requirements.\+txt 
\begin{DoxyCode}
# lancer le fichier app.py depuis le terminal !
```streamlit run app.p
\end{DoxyCode}
 \section*{pour arreter l\textquotesingle{}environnement virtuel}

deactivate

\# si besoin de désinstaller des prog 
\begin{DoxyCode}
#  les templates par défaut doivent se trouver dans le repertoire à la racine de app.py dans un dossier
       templates
/my\_streamlit\_app
|-- /templates
|-- /static
|   |-- /images
|   |-- /css
|-- app.py

# sur le navigateur ou http://localhost:8501/


# TEMPLATE
## Affichage de données
Texte simple :
```bash
st.text('Affiche du texte simple.')
\end{DoxyCode}
 \#\# Markdown \+: 
\begin{DoxyCode}
st.markdown('**Markdown** \_supporté\_.')
\end{DoxyCode}
 \#\# Messages d\textquotesingle{}alerte \+: 
\begin{DoxyCode}
st.success('Succès!')
st.info('Information.')
st.warning('Attention.')
st.error('Erreur.')
\end{DoxyCode}
 \#\# Latex \+: 
\begin{DoxyCode}
st.latex(r''' e^\{i\(\backslash\)pi\} + 1 = 0 ''')
\end{DoxyCode}
 \section*{Widgets}

\#\# Bouton \+: 
\begin{DoxyCode}
if st.button('Dis Bonjour'):
    st.write('Bonjour!')
\end{DoxyCode}
 \#\# Case à cocher \+: 
\begin{DoxyCode}
if st.checkbox('Afficher/Cachez'):
    st.text('Affiché!')
\end{DoxyCode}
 \#\# Sélecteur \+: 
\begin{DoxyCode}
option = st.selectbox('Choisissez un numéro:', [1, 2, 3])
st.write(f'Vous avez sélectionné: \{option\}')
\end{DoxyCode}
 \#\# Curseur \+: 
\begin{DoxyCode}
age = st.slider('Quel est votre âge?', 0, 130, 25)
st.write(f"J'ai \{age\} ans")
\end{DoxyCode}
 \section*{Affichage de données complexes}

\#\# Data\+Frames \+: 
\begin{DoxyCode}
import pandas as pd

df = pd.DataFrame(\{'col1': [1, 2, 3], 'col2': [4, 5, 6]\})
st.dataframe(df)
\end{DoxyCode}
 \#\# Graphiques \+: 
\begin{DoxyCode}
import matplotlib.pyplot as plt
import numpy as np

fig, ax = plt.subplots()
ax.hist(np.random.randn(1000), bins=30)
st.pyplot(fig)
\end{DoxyCode}
 \section*{Mise en cache}

\subsection*{Utilisez le décorateur .cache pour mettre en cache le résultat d\textquotesingle{}une fonction coûteuse \+:}


\begin{DoxyCode}
import time

@st.cache
def fonction\_coûteuse(param):
    # Simulation d'une fonction coûteuse
    time.sleep(5)
    return param * 10

resultat = fonction\_coûteuse(5)
st.write(resultat)
\end{DoxyCode}
 \section*{Disposition}

\#\# Colonnes \+: 
\begin{DoxyCode}
col1, col2 = st.columns(2)

with col1:
    st.header('Colonne 1')
    st.write('Quelque chose ici')

with col2:
    st.header('Colonne 2')
    st.write('Quelque chose là')
\end{DoxyCode}
 \subsection*{Expander \+:}


\begin{DoxyCode}
with st.expander("Voir détails"):
    st.write("Des détails cachés ici.")
\end{DoxyCode}
 