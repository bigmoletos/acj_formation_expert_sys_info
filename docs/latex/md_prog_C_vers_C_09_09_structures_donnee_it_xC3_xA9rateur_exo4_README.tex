A. Vérifier si un nombre est négatif dans une liste B. Tri par valeur absolue

\subsection*{Étape 1 \+: Définition et Utilisation Simple}


\begin{DoxyEnumerate}
\item Déclarez une lambda qui affiche {\ttfamily \char`\"{}\+Bonjour tout le monde !\char`\"{}}.
\item Déclarez une lambda qui prend deux entiers en paramètres et retourne leur produit.
\item Testez les deux lambdas en les appelant depuis votre fonction {\ttfamily main}.
\end{DoxyEnumerate}

\subsection*{Étape 2 \+: Utilisation avec les Conteneurs}


\begin{DoxyEnumerate}
\item Créez un {\ttfamily std\+::vector} contenant les entiers de 1 à 10.
\item Utilisez {\ttfamily std\+::for\+\_\+each} et une lambda pour afficher chaque élément du vector.
\item Modifiez chaque élément du vector en le multipliant par 3 à l’aide d’une lambda.
\end{DoxyEnumerate}

\subsection*{Étape 3 \+: Captures dans une Lambda}


\begin{DoxyEnumerate}
\item Déclarez une variable {\ttfamily factor} et affectez-\/lui la valeur {\ttfamily 4}.
\item Créez une lambda qui capture {\ttfamily factor} par {\bfseries valeur} et retourne le produit d’un entier donné avec {\ttfamily factor}.
\item Modifiez la valeur de {\ttfamily factor} après la déclaration de la lambda et vérifiez si la valeur utilisée par la lambda change.
\end{DoxyEnumerate}

\subsection*{Étape 4 \+: Captures par Référence}


\begin{DoxyEnumerate}
\item Déclarez une variable {\ttfamily factor} et affectez-\/lui la valeur {\ttfamily 4}.
\item Créez une lambda qui capture {\ttfamily factor} par {\bfseries référence} et retourne le produit d’un entier donné avec {\ttfamily factor}.
\item Modifiez la valeur de {\ttfamily factor} après la déclaration de la lambda et observez si la valeur utilisée par la lambda change.
\end{DoxyEnumerate}

\subsection*{Étape 5 \+: Captures Générales}


\begin{DoxyEnumerate}
\item Déclarez deux variables {\ttfamily x} et {\ttfamily y} initialisées à 3 et 5 respectivement.
\item Créez une lambda qui capture {\bfseries toutes les variables} par valeur et retourne leur somme.
\item Créez une autre lambda qui capture {\bfseries toutes les variables} par référence et modifie leur valeur.
\end{DoxyEnumerate}

\subsection*{Étape 6 \+: Utilisation Avancée avec la S\+TL}


\begin{DoxyEnumerate}
\item Créez un {\ttfamily std\+::vector} contenant les entiers de 1 à 10.
\item Utilisez une lambda avec {\ttfamily std\+::find\+\_\+if} pour trouver le premier élément supérieur à 5.
\item Utilisez une lambda avec {\ttfamily std\+::count\+\_\+if} pour compter les éléments pairs.
\end{DoxyEnumerate}

\subsection*{Étape 7 \+: Lambdas avec {\ttfamily std\+::function}}


\begin{DoxyEnumerate}
\item Déclarez une {\ttfamily std\+::function} qui encapsule une lambda pour calculer la différence entre deux entiers.
\item Changez dynamiquement la fonction pour encapsuler une lambda qui calcule le produit.
\end{DoxyEnumerate}

\subsection*{Étape 8 \+: Lambdas Génériques}


\begin{DoxyEnumerate}
\item Créez une lambda générique qui retourne le maximum entre deux valeurs.
\item Testez votre lambda avec des types différents ({\ttfamily int}, {\ttfamily double}, etc.). 
\end{DoxyEnumerate}