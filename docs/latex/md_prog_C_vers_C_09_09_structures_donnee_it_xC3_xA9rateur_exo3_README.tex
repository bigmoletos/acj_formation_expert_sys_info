Vous devez tester si une chaîne de caractères est un palindrome, c’est-\/à-\/dire un mot pouvant être lu indifféremment de gauche à droite ou de droite à gauche. Un exemple \+: “kayak”.

\section*{Split \href{https://zestedesavoir.com/tutoriels/822/la-programmation-en-c-moderne/le-debut-du-voyage/deployons-la-toute-puissance-des-conteneurs/#couper-une-cha%C3%AEne}{\tt https\+://zestedesavoir.\+com/tutoriels/822/la-\/programmation-\/en-\/c-\/moderne/le-\/debut-\/du-\/voyage/deployons-\/la-\/toute-\/puissance-\/des-\/conteneurs/\#couper-\/une-\/cha\%\+C3\%\+A\+Ene}}

Une autre opération courante, et qui est fournie nativement dans d’autres langages comme Python ou C\#, consiste à découper une chaîne de caractères selon un caractère donné. Ainsi, si je coupe la chaîne \char`\"{}\+Salut, ceci est une phrase.\char`\"{} en fonction des espaces, j’obtiens en résultat \mbox{[}\char`\"{}\+Salut,\char`\"{}, \char`\"{}ceci\char`\"{}, \char`\"{}est\char`\"{}, \char`\"{}une\char`\"{}, "phrase.\mbox{]} 