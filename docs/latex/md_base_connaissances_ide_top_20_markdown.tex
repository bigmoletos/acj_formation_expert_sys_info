Voici le top 20 des syntaxes Markdown les plus couramment utilisées \+:

{\bfseries Titres} \+: Utilisez {\ttfamily \#} pour les titres. Par exemple, {\ttfamily \# Titre 1}, {\ttfamily \#\# Titre 2}, jusqu\textquotesingle{}à {\ttfamily \#\#\#\#\#\# Titre 6}¹.

{\bfseries Paragraphe} \+: Pour créer un nouveau paragraphe, laissez simplement une ligne vide entre deux portions de texte¹.

{\bfseries Lien} \+: La syntaxe pour les liens est {\ttfamily \mbox{[}Texte du lien\mbox{]}(url)}¹.

{\bfseries Gras} \+: Pour mettre du texte en gras, utilisez deux astérisques ou deux tirets bas. Par exemple, {\ttfamily $\ast$$\ast$texte en gras$\ast$$\ast$} ou {\ttfamily \+\_\+\+\_\+texte en gras\+\_\+\+\_\+}⁵.

{\bfseries Italique} \+: Pour mettre du texte en italique, utilisez un astérisque ou un tiret bas. Par exemple, {\ttfamily $\ast$texte en italique$\ast$} ou {\ttfamily \+\_\+texte en italique\+\_\+}⁵.

{\bfseries Gras et italique} \+: Pour mettre du texte en gras et en italique, utilisez trois astérisques ou trois tirets bas. Par exemple, {\ttfamily $\ast$$\ast$$\ast$texte en gras et italique$\ast$$\ast$$\ast$} ou {\ttfamily \+\_\+\+\_\+\+\_\+texte en gras et italique\+\_\+\+\_\+\+\_\+}⁵.

{\bfseries Barré} \+: Pour barrer du texte, utilisez deux tildes. Par exemple, {\ttfamily $\sim$$\sim$texte barré$\sim$$\sim$}¹.

{\bfseries Citation} \+: Pour créer une citation, utilisez le signe {\ttfamily $>$}¹.

{\bfseries Liste non ordonnée} \+: Pour créer une liste non ordonnée, utilisez {\ttfamily $\ast$}, {\ttfamily -\/}, ou {\ttfamily +}¹.

{\bfseries Liste ordonnée} \+: Pour créer une liste ordonnée, utilisez des nombres suivis de {\ttfamily .}¹.

{\bfseries Image} \+: La syntaxe pour les images est {\ttfamily !\mbox{[}Texte alternatif\mbox{]}(url)}¹.

{\bfseries Code} \+: Pour afficher du code, utilisez une apostrophe inversée (backtick) `⁵.

{\bfseries Bloc de code} \+: Pour afficher un bloc de code, utilisez trois apostrophes inversées (backticks) ⁵.

{\bfseries Tableau} \+: Pour créer un tableau, utilisez des barres verticales {\ttfamily $\vert$} et des tirets {\ttfamily -\/}⁵.

{\bfseries Saut de ligne} \+: Pour créer un saut de ligne, utilisez deux espaces ou plus à la fin d\textquotesingle{}une ligne⁵.

{\bfseries Lien de référence} \+: Pour créer un lien de référence, utilisez cette syntaxe {\ttfamily \mbox{[}Texte du lien\mbox{]}\mbox{[}identifiant\mbox{]}} et définissez l\textquotesingle{}identifiant ailleurs dans le document comme ceci {\ttfamily \mbox{[}identifiant\mbox{]}\+: url}⁵.

{\bfseries Image de référence} \+: Pour créer une image de référence, utilisez cette syntaxe {\ttfamily !\mbox{[}Texte alternatif\mbox{]}\mbox{[}identifiant\mbox{]}} et définissez l\textquotesingle{}identifiant ailleurs dans le document comme ceci {\ttfamily \mbox{[}identifiant\mbox{]}\+: url}⁵. XX {\bfseries Note de bas de page} \+: Pour créer une note de bas de page, utilisez cette syntaxe {\ttfamily \mbox{[}$^\wedge$identifiant\mbox{]}} et définissez l\textquotesingle{}identifiant ailleurs dans le document comme ceci {\ttfamily \mbox{[}$^\wedge$identifiant\mbox{]}\+: Note de bas de page}⁵.

{\bfseries Ligne horizontale} \+: Pour créer une ligne horizontale, utilisez trois astérisques {\ttfamily $\ast$$\ast$$\ast$}, trois tirets {\ttfamily -\/-\/-\/} ou trois tirets bas {\ttfamily \+\_\+\+\_\+\+\_\+}⁵. 37. {\bfseries Échappement} \+: Pour afficher des caractères littéraux normalement utilisés dans la syntaxe Markdown, utilisez le caractère d\textquotesingle{}échappement {\ttfamily \textbackslash{}}⁵. 